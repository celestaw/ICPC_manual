% !TEX program = platex
\documentclass[a4paper,12pt]{jsarticle}
\usepackage{amsmath,amsthm,amssymb}
\usepackage{enumitem}
\usepackage[dvipdfmx]{graphicx}
\usepackage{placeins}

% 定理環境
\theoremstyle{definition}
\newtheorem{definition}{定義}[section]
\theoremstyle{plain}
\newtheorem{theorem}[definition]{定理}
\newtheorem{lemma}[definition]{補題}

\title{ICPC開催マニュアル}
\author{}
\date{}

\begin{document}
\maketitle

\section{概要}
ICPCの開催に関するマニュアルです。随時更新予定。主に学内で開催する

\section{準備日程}
\begin{itemize}
    \item 前年度 10月~11月: 教授陣との打ち合わせ
    \item 開催年度 4月: 新入生ガイダンスでの告知
    \item 開催年度 5月末: ICPCアジア予選告知開始
    \item 開催年度 6月上旬: 授業内告知、説明会、登録会
    \item 開催年度 6月下旬: 初心者講習会、模擬戦感想会
    \item 開催年度 7月中旬: ICPCアジア予選
\end{itemize}

\subsection{10月~11月 教授陣との打ち合わせ}
ICPCの開催に関する教授陣との打ち合わせを行う。目的は情報科学科で開催する場合その承認を得ること、監督を担当する教授陣の確保の二つである。
\subsubsection{情報科学科での開催承認}
oh-o meiji!上にICPC開催の広告を載せる際や新入生ガイダンスで告知するのに必要。その他いろいろメリットがあるため開催承認を行うことを推奨する。成績優秀者への報償は毎度教授陣有志の寄付に頼っているため、情報科全体で開催することで一人当たりの負担が減るので好ましい。相談は学科長にする必要があり、2025年度は飯塚先生に相談する必要があった。毎年変わる可能性があるため学科長の確認はその都度とるようにする。新年度が始まるタイミングで変化することもある。教授に相談する際は、事前にメールでアポイントメントをとるとよい。話す内容は以下の通り。
\begin{itemize}
    \item ICPCの概要
    \item ガイダンスでの告知について
    \item ICPCの概要をoh-o meiji!上にお知らせすることについて
    \item 成績優秀者への報償について
\end{itemize}
なお、情報科学科で開催するにあたって、成績優秀者への報償が問題になる可能性がある。もし報償をなくさないと承認できない場合、報償をなくすことは大いに検討に値する。報償がなくてもICPCに参加する学生はいるため、開催自体は可能である。これについては開催にあたっての問題を参照。
\subsubsection{監督の確保}
ICPCの監督は教授陣が担当する。監督がいないとICPCを開催できないため、例年松田先生と宮島先生に担当してもらっている。2026年度は西郷氏からお願いすることになると思われるがその後の年度を考慮するとコーチ担当者、または進行役を務める補佐が直接頼み込みに行くのがよいと思われる。先と同様に事前にメールでアポイントメントをとることを推奨する。

補足:2026年度は学科長が飯塚先生のままなので大丈夫だと思われるが、2027年度以降は変わる可能性があるため注意。
\subsection{4月 新入生ガイダンスでの告知}
情報科学科で開催することができると新入生ガイダンスの際にICPCについて告知することが可能になる。1年生全員に周知できる機会でありかなり貴重。この確認は情報科学科での開催承認の時に話すときにガイダンスでの告知の許可をもらい、後日パワポ資料をメールで送る形にする.このとき,原稿を渡して読んでもらってもいいし、自分が説明しに行ってもよい。

\subsection{5月末 ICPCアジア予選告知開始}
5月末頃、
\section{役割分担}
\subsection{コーチ}
学生が担当する必要がある。主にICPCの運営、学生のサポートを行う。

\subsection{監督}
教授陣が担当する。問題回答中の不正行為の監視、運営への報告を行う。

\subsubsection{コーチ補佐}
ICPCに割り振られる役割ではない。コーチの仕事を補佐する。ICPCの選手にならない限り試験中の部屋に入ることはできない。

\section{開催にあたっての問題}
\subsection{成績優秀者への報償}
情報科学科で開催することに対して最大の壁になるものである。基本的に教授全体から寄付を募れることはないとみてよい(こちらから直接頼み込みに行く必要はない)。ただし情報科主体の開催そのものと寄付を全体からもらうことはあくまで別問題であるため特段全員から寄付を募らなくてもよい。なお寄付を渋る教授陣の意見としては情報科学科でない学生に報償がわたってしまうことが挙げられている。しかしこの他学科からの参加はこのICPCの大会を開くにあたって大いに歓迎するものであって、仮にこれを情報科学科の生徒だけに絞るのは開催の意図に反してしまう。このためこの問題は基本解決できず、今後もあくまで有志からの寄付に頼ることになると思われる。

\end{document}